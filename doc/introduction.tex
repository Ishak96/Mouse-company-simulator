\section{Introduction}
\label{sec:introduction}

\paragraph{Contexte:}
Dans le cadre de nos études en licence d'informatique, nous devons réaliser en trinôme, un projet de programmation Java pour le module de génie logiciel. Nous avons choisi le projet des souris coopératives car il est trés intéressant pour nous, d'étudier le concept de la communication, la gestion de mémoire et l'intelligence artificielle.

\paragraph{Objet:}
Créer une simulation où des souris évoluent sur une grille, en se nourrissant et en se communiquant.

\paragraph{Outils de développement:}
Nos outils de développement sont ceux qui nous ont été conseillés par notre enseignant, et que nous avons utilisé au cours de ce semèstre en Génie Logiciel et Projet. Nous avons utilisé la plateforme Java 9 ainsi que l'environnement de développement Eclipse. Nous avons synchronisé notre travail en utilisant le service web d'hébergement GitHub qui nous a beaucoup aidé pour tavailler en groupe. Enfin, ce rapport de projet a été rédigé avec LaTex sur le service web Overleaf.

\paragraph{Structure du rapport}:
Notre première partie concernera les spécifications de notre projet, elle contiendra toutes les fonctionnalités du logiciel. Ensuite nous parlerons de la partie réalisation, dans laquelle nous présenterons la conception détaillée de notre logiciel, et viendra après le manuel utilisateur où le fonctionnement sera expliqué. Puis nous aborderons la partie déroulement où le calendrier de notre travail et la répartition des tâches seront présentés, pour arriver ensuite à la conclusion où nous allons donner le bilan de notre projet.